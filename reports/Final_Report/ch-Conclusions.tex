\setcounter{chapter}{6}
\setcounter{section}{0}
\setcounter{subsection}{0}
\chapter*{Conclusions}
\addcontentsline{toc}{chapter}{Conclusions}
The success of the MARIA, Wavelia and TSAR clinical trials indicate that microwave imaging can provide a viable
alternative to traditional X-ray and ultrasound modalities. Aside from quantitative metrics, the subjective experiences
of women also play a significant role in the success of an imaging modality. One qualitative study on women's
satisfaction after receiving a mammogram found that they felt pain and discomfort during the imaging process, with one
woman stating that they never showed up to their second appointment due to the pain experienced in the first one
\cite{engelmanWomenSatisfactionTheir2006}. Whereas a survey conducted on the participants of the Wavelia clinical trial
showed that 92\% of the women found the bed comfortable and would recommend the procedure to other women. Indicating
that women feel safe and comfortable with this new imaging modality. While the hardware has progressed, generalized
software that can analyze the data from these systems has been lacking. Before the MERIT library created
by Dr. O'Loughlin for the MATLAB programming language, most available software was specific to a particular system or created to answer a specific research
question as was discussed at length in section \ref{MERITAsLibrary}. MERIT.jl was motivated by this lack of software and
was created to be an alternative for researchers who want to use the Julia programming language in their studies.

MERIT was used to benchmark the MERIT.jl library in terms of performance and results. By making use of features such as
type stability, MERIT.jl can process the data from the MERIT tutorial in under 12 seconds when running on a single
thread on an i7-1185G7. MERIT.jl also demonstrates a similar level of customizability and flexibility as the MATLAB
library by allowing for custom delay and beamforming functions, as well as allowing researchers to easily change the
$\varepsilon$ parameter in the default delay function. MERIT.jl also offers a form of reliability that is not seen in
the MATLAB library through the introduction of a Point data type. This Point type allowed for a distinction to be
created between a collection of numbers that represent a coordinate point, and a general collection of numbers. If the
user was to incorrectly pass a collection of the point type to a function in place of another argument, the Julia
compiler would throw an error, thereby creating a level of reliability. MERIT.jl also makes use of the multiple dispatch
feature in Julia to hide implementation-specific details from the end users. This is heavily used in the Point
type to allow the inbuilt mathematical operators to work on the Point type. This feature could also be used when the
library needs to be extended to include different body parts. If the library needed to accommodate the data from the
study conducted by Zhao \textit{et al.} \cite{gilmoreMicrowaveImagingHuman2013}, one would only have to create a
ForearmScan struct subtyped from the Scan abstract struct, and then overload the domain generation function. Users
following the current pipeline in MERIT.jl would not notice any difference as the compiler will handle dispatching the
correct functions. This demonstrates another one of the benefits that MERIT.jl has over the MATLAB version. The only way
in MATLAB to achieve this would be to create long chains of if-else statements that check the types of the provided
input. Whereas in Julia, each specific implementation of a function can be grouped under one name. In this way,
functions in MERIT.jl can be considered as concepts that encapsulate functional implementations. As was seen in section
\ref{PlottedScans}, the MERIT.jl library provides numerically identical outputs to the MATLAB implementation with an
averaged MSE of $8.4417 \times 10^{-7}$, showing that the MERIT.jl library can be a viable alternative to the MATLAB
library.

The Julia programming language offers many beneficial features for developers creating libraries for scientific
computing. The JIT compiler and type stability allow developers to write high-level code which is then optimized and
compiled at runtime, simplifying the development process. Efficient garbage collection means that developers no longer
have to worry about deallocating assigned memory, removing the risk of inadvertently causing memory leaks.
%%Ability to call other languages
%%Ability to use + extend other libraries
%%ABility to use mathmatical notation
%%multiple dispatch 