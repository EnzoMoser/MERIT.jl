\chapter*{Introduction}
\addcontentsline{toc}{chapter}{Introduction}
MERIT.jl was motivated by a need to streamline the development process for new imaging algorithms. Currently, any
researcher who wants to code a novel microwave imaging algorithm not only has to code the algorithm itself but also has
to code all the helper functions required to produce an image. All this time fixing bugs subtracts from the time that could
be spent fine-tuning the algorithm and choosing an optimal parameter set. The use of such open-source software has seen
widespread use in a myriad of fields. PsychoPy and PsyToolkit are two such frameworks that revolutionized the field of
psychological sciences. By implementing commonly used functions and scripts, they have allowed researchers to design and
run experiments in a matter of hours. It has also allowed researchers who have little to no programming experience
to get up and running with automated data processing, thereby allowing them to focus more on the quality of their
experiment. \cite{stoetPsyToolkitTestimonials} \hfill \break

The use of microwaves in imaging has started to gain interest amongst the medical community as an alternative and safer
form of imaging when compared to more traditional methods such as X-rays. Clinical trials such as the ``MARIA M4''
system has proven that such microwave-based methods are more comfortable and are a viable alternative to current
mammograms \cite{preeceMARIAM4Clinical2016}. Mammography is not the only area where this imaging modality is being
trailed. It is seeing use in areas such as traumatic brain injury detection, bone degradation and tumor detection
\cite{alsbouMedicalImagingSystem2023}. While the hardware has proved effective, the software leaves a lot to be desired.
\hfill \break

That is where this project comes in. MERIT.jl aims to be an easy-to-use, extensible and featureful library. The goal
being that anyone, regardless of coding experience, would be able to quickly create a script that allows them to process
and visualize the scan data they have collected. The following sections will contain: 

\begin{itemize}
    \item A literature review on the existing microwave imaging systems
    \item A look into existing microwave imaging frameworks
    \item A discussion about the design choices and Julia features that are included
    \item An examination of the results and possible future work
\end{itemize}

