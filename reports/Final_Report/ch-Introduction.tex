\chapter*{Introduction}
\addcontentsline{toc}{chapter}{Introduction}
MERIT.jl was motivated by a need to increase the interoperability between the data produced from systems and the
programs that would be able to process them. Currently, any researcher who wants to design a novel microwave imaging
algorithm or test the efficacy of their new systems has to spend time and effort creating entirely new software suites.
More often than not these programs either tend to be closed source, or not compatible with other systems mainly because
there are not many incentives for a researcher to invest the time required in creating a truly configurable and
extensible open-source system. With the rise and success of clinical trials in microwave imaging, this problem is only
going to get worse as more and more systems are developed.  \hfill \break

The use of such open-source software has seen widespread use in a myriad of fields. PsychoPy and PsyToolkit are two such
frameworks that revolutionized the field of psychological sciences. By implementing commonly used functions and scripts,
they have allowed researchers to design and run experiments in a matter of hours. It has also allowed researchers who
have little to no programming experience to get up and running with automated data processing, thereby allowing them to
focus more on the quality of their experiment \cite{stoetPsyToolkitTestimonials}. \hfill \break

That is where this project comes in. MERIT.jl aims to be an easy-to-use, extensible and featureful library. The goal
being that anyone, regardless of coding experience, would be able to quickly create a script that allows them to process
and visualize the scan data they have collected, as well as allowing more experienced coders to develop and test new
algorithms with ease. The following sections will contain: 

\begin{itemize}
    \item A literature review on the existing microwave imaging systems
    \item A look into existing microwave imaging frameworks
    \item A discussion about the design choices and Julia features that are included
    \item An examination of the results and possible future work
\end{itemize}

The source code and the library are hosted on GitHub for anyone to view and amend:
\url{https://github.com/AaronDinesh/MERIT.jl}
