\chapter*{Introduction}
\addcontentsline{toc}{chapter}{Introduction}
Microwave imaging has seen a rising interest in the medical field evidenced by the numerous clinical trials that are
being conducted by research groups around the world \cite{preeceMARIAM4Clinical2016, moloneyWaveliaMicrowaveBreast2021,
e.c.fearMicrowaveBreastImaging2013}. A cursory search on GitHub for software around microwave imaging yielded few useful
results, with many being specialized repositories for a particular task or performing some machine learning analysis on
microwave data. Only one repository stood out as a generalized library that provides researchers with all the tools
needed to easily test different algorithms; the MERIT toolbox developed by Prof O'Loughlin, M. A. Elahi, E. Porter, et.
al \cite{d.oloughlinOpensourceSoftwareMicrowave2018}. Other fields of research have seen numerous benefits from the
introduction of comprehensive open-source libraries. Libraries such as PsychoPy and PsyToolkit have allowed psychology
researchers to design and conduct experiments in a matter of hours by packaging common functions in an easy-to-use
library \cite{stoetPsyToolkitTestimonials}. With a rise in the number of systems that can perform microwave imaging
and the vast amounts of data being generated from these systems, it is imperative that there are a variety of toolkits
available to not only analyze data from these current systems but also from future systems. This BAI project aims to
consider the following questions:
\begin{itemize}
    \item Improving the accessibility of Microwave Imaging
    \item Increasing the compatibility between systems, the data these systems generate and the software used to analyze
    this data
    \item Creating an intuitive, easily extensible and customizable library
    \item Leveraging the features of a coding language to create a performant library   
\end{itemize}
\hfill \break

The rest of the report will be divided up as follows:
\begin{itemize}
    \item A literature review on existing microwave imaging systems
    \item A look into existing reconstruction algorithms 
    \item A discussion about the design choices and Julia features that are included
    \item An examination of the results and possible future work
\end{itemize}

MERIT.jl being an open-source library has all its code available on GitHub for anyone to view and amend. It can be
viewed at the following URL: \url{https://github.com/AaronDinesh/MERIT.jl}
