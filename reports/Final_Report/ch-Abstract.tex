\chapter*{Abstract}
Microwave imaging is being trialed as a safer and cheaper imaging modality over X-rays and ultrasound which are the
current standard in medical imaging. In the last 8 years, 10 clinical studies have been conducted that evaluated novel
microwave imaging systems, focused mainly in the area of breast tumor detection. The success of these trials has proven
the viability of microwaves as an alternate imaging modality. MERIT.jl, then, aims to be a reliable, user-friendly and
high-performance software framework that aims to increase the interoperability between microwave imaging systems, the
data they produce and the software that analyzes that data. It contains helper functions as well as inverse scattering
algorithms to aid in image reconstruction. The unique features in Julia, such as parametric polymorphism, multiple
dispatch and type stability to name a few, have allowed MERIT.jl to possess unparalleled customizability and
extensibility. MERIT.jl also demonstrates a level of type safety by implementing a lightweight Points type that allows
for the efficient processing of coordinate points. Unlike traditional methods that represent points as a matrix of
numbers, the Points type allows the Julia compiler to recognize when an incorrect collection is passed in place of the
points collection, throwing an error in the process. In this way, MERIT.jl can promise a level of safety without having
to compromise on performance. MERIT.jl is significant as it is the first generalized medical microwave imaging toolbox
written for the Julia language and ecosystem. 
\newpage
