\chapter{Introduction}
MERIT.jl was motivated by a need to streamline the development process for new imaging algorithms. Currently, any
researcher who wants to code a novel microwave imaging algorithm not only has to code the algorithm itself but also has
to code all the helper functions required to produce an image. All this time fixing bugs subtracts from the time that could
be spent fine-tuning the algorithm and choosing an optimal parameter set. The use of such open-source software has seen
widespread use in a myriad of fields. PsychoPy and PsyToolkit are two such frameworks that revolutionized the field of
psychological sciences. By implementing commonly used functions and scripts, they have allowed researchers to design and
run experiments in a matter of hours. It has also allowed researchers who have little to no programming experience
to get up and running with automated data processing, thereby allowing them to focus more on the quality of their
experiment. \cite{stoetPsyToolkitTestimonials} \hfill \break

The use of microwaves in imaging has started to gain interest amongst the medical community as an alternative and safer
form of imaging when compared to more traditional methods such as X-rays. Clinical trials such as the ``MARIA M4''
system, have proven that such microwave-based methods are more comfortable and are a viable alternative to current
mammograms.   






####################### Remove #######################
The use of such OS frameworks has seen much success in other fields such as PsychoPy and PsyToolkit in the psychological
sciences, and Maxima in mathematical computing. The groundwork was laid by D. O’Loughlin, M. A. Elahi, E. Porter, et al
in their paper entitled “Open-source Software for Microwave Radar-based Image Reconstruction”. Here they implemented a
series of classic beamformers such as “Delay-and-Sum”, “Modified-DAS” and “Multiply-DAS”. The MATLAB library also
provides a series of utility functions that can be called to generate the imaging domain, or to generate an image slice
from the beamformed data, for example. 

The main drawback of MATLAB is that the language is far from performant, running about 2.24x slower than an equivalent
Julia script [1]. Julia implements the “Multiple Dispatch” programming paradigm, which allows for powerful extensibility
and an intuitive interface.

\section*{Introduction}
Microwave imaging has started to garner an interest amongst the medical community as an alternative and safer form of
imaging when compared to more traditional methods such as X-rays. Recent clinical trials such as the “MARIA M4” system
have proven that such microwave-based methods are more comfortable and offer a viable alternative to X-rays \cite{RN1}.
Mammography is not the only area where this imaging modality is being trialed. It is seeing use in areas such as
traumatic brain injury detection, bone degradation, and tumor detection \cite{RN2}. While the hardware has proved
effective, the software side leaves a lot to be desired. Researchers often have to code their own data processing
pipeline in order to get usable results for their studies. All this development subtracts from time that could be spent
developing new algorithms and higher resolution systems. In their haste to complete a paper, bugs could be inadvertently
introduced into the code, at best slowing down research while the bug is fixed, while at worst, it could bias the
results without the researchers knowing. As such, good software needs to be built that would allow researchers to worry
more about designing and testing algorithms rather than worrying about how to code the supporting functions that would
allow them to test the effectiveness of these algorithms.